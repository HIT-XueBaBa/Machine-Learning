\documentclass{article}
\usepackage{CJKutf8}
\usepackage[namelimits]{amsmath} %数学公式
\usepackage{amssymb}             %数学公式
\usepackage{amsfonts}            %数学字体
\usepackage{mathrsfs}            %数学花体
\usepackage[linesnumbered,boxed,ruled,commentsnumbered]{algorithm2e}%%算法包

\newtheorem{myDef}{Definition} 

\title{Prototype-based clustering}
\author{Chong Liu\qquad Harbin Institute of Technology}
 
\usepackage{indentfirst}
\setlength{\parindent}{2em}

\numberwithin{equation}{section}

\begin{document}

\begin{CJK*}{UTF8}{gbsn}

\maketitle

\tableofcontents
\newpage
\section{前言}
原型聚类亦称“基于原型的聚类”,此类算法假设聚类结构能通过一组原型刻画,在现实聚类任务中极为常用。通常情况下,算法先对原型初始化,然后对原型进行迭代更新求解。

\section{k均值算法}
给定样本集$D=\{x_1,x_2,\ldots,x_m\}$,“k均值”(k-means)算法针对聚类所得簇划分$C=\{C_1,C_2,\ldots,C_k\}$最小化平方误差$$E=\sum_{i=1}^k\sum_{x\in{C_i}}{\begin{Vmatrix}x-\mu_i\end{Vmatrix}}_2^2\eqno{(1)}$$其中$\mu_i=\frac{1}{|C_i|}\sum_{x\in{C_i}}x$是簇$C_i$的均值向量。直观看来,式(1)在一定程度上刻画了簇内样本围绕簇均值向量的紧密程度,$E$值越小则簇内样本相似度越高。
\par
最小化式(1)并不容易,找到它的最优解需考察样本集$D$所有可能的簇划分,这是一个NP难的问题[Aloise et al., 2009],因此,k均值算法采用了贪心策略,通过迭代优化来近似求解式(1)。算法流程如算法1所示,其中第1行对均值向量进行初始化,在第4—8行与第9—16行依次对当前簇划分及均值向量迭代更新,若迭代更新后聚类结果保持不变,则在第18行将当前簇划分结果返回。
\par
\begin{algorithm}[H]
\SetAlgoNoLine
\caption{k均值算法}
\LinesNumbered
\KwIn{样本集$D=\{x_1,x_2,\ldots,x_m\}$和聚类簇数k}
从$D$中随机选择k个样本作为初始均值向量$\{\mu_1,\mu_2,\ldots,\mu_k\}$\\
\Repeat{当前均值向量均未更新}{
    令$C_i = \phi(1\leq i\le k)$\\
    \For{$j=1,,\ldots,m$}{
        计算样本$x_j$与各均值向量$\mu_i(1\le i\le k)$的距离:$d_{ji}=\|x_j-\mu_i\|_2$\;
        根据距离最近的均值向量确定$x_j$的簇标记:$\lambda_j=arg \quad min_{i\in \{1,2,\ldots,k\}}\quad{d_{ji}}$\;
        将样本$x$划入相应的簇:$C_{\lambda_j}\bigcup{x_j}$;
    }
    \For{$i=1,2,\ldots,k$}{
        计算新均值向量:$\mu_i'=\frac{1}{|C_i|}\sum_{x\in{C_i}}x$\;
        \uIf{$\mu_i'\neq \mu_i$}{
        将当前均值向量$\mu_i$更新为$\mu_i'$
        }
        \Else{保持当前均值向量不变\\}
    }
}
\KwOut{簇划分$C=\{C_1,C_2,\ldots,C_k\}$}
\end{algorithm}

\section{EM算法}
未观测变量的学名式“隐变量”(latent variable)。令$X$表示已观测变量集,\\$Z$表示隐变量集,$\Theta$表示模型参数。若欲对$\Theta$做极大似然估计,则应最大化对数似然$$LL(\Theta|X,Z)=lnP(X,Z|\Theta)\eqno{(2)}$$然而由于$Z$是隐变量,上式无法直接求解。此时我们可通过对$Z$计算期望,来最大化已观测数据的对数“边际似然”(marginal likelihood)$$LL(\Theta|X)=lnP(X|\Theta)=ln\sum_ZP(X,Z|\Theta)\eqno{(3)}$$
\par
EM(Expectation-Maximization)算法[Dempster et al.,1977]是常用的估计隐变量的利器,它是一种迭代式的方法,其基本想法是:若参数$\Theta$已知,则可根据训练数据推断出最优隐变量$Z$的值(E步);反之,若$Z$的值已知,则可以方便的对参数$\Theta$做极大似然估计(M步)
\par
进一步,若我们不是取$Z$的期望,而是基于$\Theta^t$计算隐变量$Z$的概率分布\\$P(Z|X,\Theta^t)$,则EM算法的两个步骤是:
\begin{enumerate}
    \item E步(Expectation):以当前参数$\Theta^t$推断隐变量分布$P(Z|X,\Theta^t)$,并计算对数似然$LL(\Theta|X,Z)$关于$Z$的期望$$Q(\Theta|\Theta^t)=\mathbb{E}_{Z|X,\Theta^t}LL(\Theta|X,Z)\eqno{(4)}$$
    \item M步(Maximization):寻找参数最大化期望似然,即$$\Theta^{t+1}={arg\quad{max}}_\Theta\quad{Q(\Theta|\Theta^t)}\eqno{(5)}$$
\end{enumerate}

\section{高斯混合聚类}
与k均值用原型向量来刻画聚类结构不同,高斯混合(Mixture-of-Gaussian)\\聚类采用概论模型来表达聚类原型。
\par
\begin{myDef}
(多元)高斯分布\par
对$n$维样本空间$\chi$中的随机向量$x$,若$x$服从高斯分布,其概率密度函数为$$p(x)=\frac{1}{(2\pi)^\frac{n}{2}|\Sigma|^\frac{1}{2}}e^{-\frac{1}{2}(x-\mu)^T\Sigma^{-1}(x-\mu)}\eqno{(6)}$$其中$\mu$是$n$维均值向量,$\Sigma$是$n\times n$的协方差矩阵。由式(2)可看出,高斯分布完全由均值向量$\mu$和协方差矩阵$\Sigma$这两个参数确定。为了明确显示高斯分布与相应参数的依赖关系,将概率密度函数记为$p(x|\mu,\Sigma)$。
\end{myDef}
\par
我们可定义
\begin{myDef}
高斯混合分布\par
$$p_M=\sum_{i=1}^k\alpha_i\cdot p(x|\mu_i,\Sigma_i)\eqno{(7)}$$
该分布共由$k$个混合成分组成,每个混合成分对应一个高斯分布。其中$\mu_i$与$\Sigma_i$\\是第$i$个高斯混合成分的参数,而$\alpha_i>0$为相应的“混合系数”(mixture coefficient),$\sum_{i=1}^k\alpha_i=1$。
\end{myDef}
\par
假设样本的生成过程由高斯混合分布给出:首先,根据$\alpha_1,\alpha_2,\ldots,\alpha_k$定义的先验分布选择高斯混合成分,其中$\alpha_i$为选择第$i$个混合成分的概率;然后,根据被选择的混合成分的概率密度函数进行采样,从而生成相应的样本。
若训练集$D=\{x_1,x_2,\ldots,x_m\}$由上述过程生成,令随机变量$z_j\in{\{1,2,\ldots,k\}}$表示生成样本$x_i$的高斯混合成分,其取值未知。显然,$z_j$的先验概率$P(z_j=i)$对应于$\alpha_i(i=1,2,\ldots,k)$。根据贝叶斯定理,$z_j$的后验分布对应于$$p_M(z_j=i|x_j)=\frac{P(z_j=i)\cdot{p_M(x_j|z_j=i)}}{p_M(x_j)}=\frac{\alpha_i\cdot{p(x_j|\mu_i,\Sigma_i)}}{\sum_{l=1}^k\alpha_l\cdot{p(x_j|\mu_l,\Sigma_l)}}\eqno{(8)}$$换言之,$p_M(z_j=i|x_j)$给出了样本$x_j$由第$i$个高斯混合成分生成的后验概率,为方便叙述,将其简记为$\gamma_{ji}(i=1,2,\ldots,k)$。
\par
当高斯混合分布(7)已知时,高斯混合聚类将把样本集$D$划分为$k$个簇$C=\{C_1,C_2,\ldots,C_k\}$,每个样本$x_j$的簇标记$\lambda_j$如下确定:$$\lambda_j={arg\quad max}_{i\in{(1,2,\ldots,k)}}\quad\gamma_{ji}\eqno{(9)}$$因此,从原型聚类的角度来看,高斯混合聚类是采用概率模型(高斯分布)对原型进行刻画,簇划分则由原型对应后验概率确定。

\section{EM算法求解高斯混合分布模型参数}
那么,对于式(7),模型参数$\{(\alpha_i,\mu_i,\Sigma_i)|1\le i\le k\}$如何求解呢?显然,给定样本集$D$,可采用极大似然估计,即最大化(对数)似然$$LL(D=ln(\prod_{j=1}^mp_M(x_j))=\sum_{j=1}^mln(\sum_{i=1}^k\alpha_i\cdot p(x_j|\mu_i,\Sigma_i))\eqno{(10)}$$
常采用EM算法进行迭代优化求解。
\par
若参数$\{(\alpha_i,\mu_i,\Sigma_i)|1\le i\le k\}$能使式(10)最大化,有
\begin{enumerate}
\item 
$\mu_i=\frac{\sum_{j=1}^m\gamma_{ji}x_j}{\sum_{j=1}^m\gamma_{ji}}$\;
\item
$\Sigma_i=\frac{\sum_{j=1}^m\gamma_{ji}(x_j-\mu_i')(x_j-\mu_i')^T}{\sum_{j=1}^m\gamma_{ji}}$\;
\item
$\alpha_i=\frac{\sum_{j=1}^m\gamma_{ji}}{m}$\;
\end{enumerate}
\par
高斯混合聚类算法描述如算法2所示。算法第1行对高斯混合分布的模型参数进行初始化。然后,在第2—12行基于EM算法对模型参数进行迭代更新。若EM算法的停止条件满足,则在第14-17行根据高斯混合分布确定簇划分,在第18行返回最终结果。
\par
\begin{algorithm}[H]
\SetAlgoNoLine
\caption{高斯混合聚类算法}
\LinesNumbered
\KwIn{样本集$D=\{x_1,x_2,\ldots,x_m\}$和高斯混合成分个数k}
初始化高斯混合分布的模型参数$\{(\alpha_i,\mu_i,\Sigma_i)|1\le i\le k\}$\\
\Repeat{满足停止条件}{
    \For{$j=1,,\ldots,m$}{
        根据式(8)计算$x_j$由各混合成分生成的后验概率,即
        $\gamma_{ji}=P_M(z_j=i|x_j)(1\le i\le k)$
    }
    \For{$i=1,2,\ldots,k$}{
        计算新均值向量:$\mu_i'=\frac{\sum_{j=1}^m\gamma_{ji}x_j}{\sum_{j=1}^m\gamma_{ji}}$\;
        计算新协方差矩阵:$\Sigma_i'=\frac{\sum_{j=1}^m\gamma_{ji}(x_j-\mu_i')(x_j-\mu_i')^T}{\sum_{j=1}^m\gamma_{ji}}$\;
        计算新混合系数:$\alpha_i'=\frac{\sum_{j=1}^m\gamma_{ji}}{m}$\;
    }
    将模型参数$\{(\alpha_i,\mu_i,\Sigma_i)|1\le i\le k\}$更新为$\{(\alpha_i',\mu_i',\Sigma_i')|1\le i\le k\}$
}
$C_i=\phi (1\le i\le k)$\\
\For{$j=1,,\ldots,m$}{
    根据式(3)确定$x_j$的簇标记$\lambda_j$\;
    将$x_j$划入相应的簇:$C_{\lambda_j}=C_{\lambda_j}\bigcup{x_j}$
}
\KwOut{簇划分$C=\{C_1,C_2,\ldots,C_k\}$}
\end{algorithm}

\section{参考文献}
[1] 周志华. 机器学习[M]. 北京:清华大学出版社, 2016.

\newpage
\end{CJK*} 
\end{document}